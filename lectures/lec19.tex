\section{Laurentzerlegung}\lecture
		
		Sei im Folgenden $ 0 \leq r < R \leq \infty $. Betrachte das Ringgebiet $$ \mathcal{R} := \{z \in \C \mid r < |z| < R\} $$
		und z.B. $ g: B_R(0) \to \C,\ h: B_{\frac{1}{r}}(0) \to \C. $ Dann ist $ \C \setminus \overbar{B_r(0)} \to \C, z \mapsto h\left(\frac{1}{z}\right) $ holomorph. Setze $ f: \mathcal{R} \to \C,\ f(z) = g(z) + h\left(\frac{1}{z}\right). $ $f$ ist holomorph auf $\mathcal{R}$. Der folgende Satz zeigt, dass jede holomorphe Funktion auf $ \mathcal{R} $ sich in dieser Weise zerlegen lässt.
		
		\begin{thmn}[Laurentzerlegung]\label{3.3.1}
			Sei $ 0 \leq r < R \leq \infty $. Jede auf dem Ringgebiet $ \mathcal{R} := \{z \in \C \mid r < |z| < R\} $ holomorphe Abbildung kann geschrieben werden als
			\begin{equation}\label{laurent}
			f(z) = g(z) + h\left(\frac{1}{z}\right)
			\end{equation}
			mit $ g: B_R(0) \to \C,\ h: B_{\frac{1}{r}}(0) \to \C $ holomorph. Fordert man noch $h(0) = 0$, so ist diese Zerlegung eindeutig bestimmt.
		\end{thmn}
		
		\begin{defn}[Hauptteil, Nebenteil, Laurentzerlegung]
			In der Situation von Satz \ref{3.3.1} ist $ z \to h\left(\frac{1}{z}\right) $ der \emph{Hauptteil} der Funktion $f$. $g$ ist der \emph{Nebenteil} der Funktion $f$ und \ref{laurent} ist die \emph{Laurentzerlegung} der Funktion $f$.
		\end{defn}
		
		\begin{lem}
			Seien $ 0 \leq r < R \leq \infty $, und sei $ \mathcal{R} := \{z \in \C \mid r < |z| < R\} $. Sei $ G: \mathcal{R} \to \C $ eine holomorphe Abbildung. Sind $ P, \rho \in \R $, sodass $ r < \rho < P < R $, dann gilt 
			\[ \varointctrclockwise_{|\zeta| = \rho} G(\zeta) \d\zeta = \varointctrclockwise_{|\zeta| = P} G(\zeta)\d\zeta. \]
		\end{lem}
		
		\begin{thmn}[Laurententwicklung]\label{3.3.4}
			Die Funktion $ f: \mathcal{R} \to \C,\ \mathcal{R} := \{z \in \C \mid r < |z| < R\} $ sei holomorph mit der Laurenzerlegung $f(z) = g(z) + h\left(\frac{1}{z}\right)$. Dann kann man $g$ und $h$ in Potenzreihen entwickeln
			\[ g(z) = \sum_{n=0}^\infty a_n z^n\ \text{für } |z| < R,\quad h(z) = \sum_{n=1}^\infty b_n z^n\ \text{für } |z| < \frac{1}{r}, \] und mit $a_{-n} = b_n$ erhält man die Laurententwicklung von $f$, welche auf $\mathcal{R}$ lokal normal konvergiert: 
			\[ f(z) = \sum_{n \in \Z} a_n (z-a)^n \qquad \forall\, z \in \mathcal{R}. \]
			Außerdem gilt
			\begin{enumerate}[label={\roman*})]
				\item Diese Laurententwicklung ist eindeutig bestimmt:
				\[ a_n = \frac{1}{2\pi i} \varointctrclockwise_{|\zeta - a| = \rho} \frac{f(\zeta)}{(\zeta - a)^{n+1}}\d\zeta \qquad \forall\, n \in \Z, r < \rho < R \]
				\item Sei $ M_\rho (f) = \sup \left\{|f(\zeta)| \mid |\zeta-a| = \rho \right\} $ für $ r < \rho < R. $ Dann gilt 
				\[ |a_n| \leq \frac{M_\rho(f)}{\rho^n}, \qquad n \in \Z,\ \text{"Cauchy'sche Abschätzungsformel"} \]
			\end{enumerate}
		\end{thmn}