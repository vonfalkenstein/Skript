\section{Der Potenzreihenentwicklungssatz}\lecture
		
		\begin{thm}\label{2.4.1}
			Sei $ f: D \to \C $ eine holomorphe Abbildung. Sei $ a \in D $ und $r>0$, sodass $ B_r(a) \subseteq D $. Dann gilt $ f(z) = \sum_{n=0}^\infty a_n(z-a)^n\ \forall \, z \in B_r(a), $ wobei $ a_n = \frac{f^{(n)}(a)}{n!}\ \forall\, n \in \N. $
		\end{thm}
		
		\begin{rem}
			\begin{enumerate}
				\item[]
				\item Die Formeln $ a_n = \frac{f^{(n)}(a)}{n!} $ folgen aus der Potenzreihenentwicklung; für $ f(z) = \sum_{n=0}^\infty a_n(z-a)^n $ folgt aus Satz \ref{thm_series}. dass die Ableitungen an $a$ $ f^{(k)}(a) = k!a_k $ erfüllen, also $ a_k = \frac{f^{(k)(a)}}{k!}\ \forall\, k \in \N. $
				\item Für die Koeffizienten $a_n$ gilt nach Satz \ref{2.3.6} 
				$$ a_n = \frac{f^{(n)}(a)}{n!} = \frac{1}{2 \pi i} \int_{|\zeta-a|=\rho} \frac{f(\zeta)}{(\zeta-a)^{n+1}} \d \zeta $$
				für $ 0 < \rho < R $.
				\item Der Satz sagt, dass holomorphe Abbildungen genau die Funktionen sind, welche sich lokal in Potenzreihen mit positivem Konvergenzradius entwickeln lassen! Daher sagt man auch "analytisch" für "holomorph".
				\item Die Koeffizienten $a_n$ sind die \emph{Taylorkoeffizienten von $f$ zur Stelle $a$}, und die Potenzreihe ist die \emph{Taylorreihe von $f$ zur Stelle $a$}.
				\item Sei $ f: \C \to \C $ eine ganze Abbildung. Dann ist nach Satz \ref{2.4.1} $ f(z) = \sum_{n=0}^\infty \frac{f^{(n)}(0)}{n!} z^n\ \forall z \in \C, $ oder allgemeiner:
				\[ f(z) = \sum_{n=0}^\infty \frac{f^{(n)}(a)}{n!} (z-a)^n \]
				für $a \in \C$ und alle $z \in \C$.
			\end{enumerate}
		\end{rem}